\documentclass[12pt,oneside]{article}

% This package simply sets the margins to be 1 inch.
\usepackage[margin=1in]{geometry}

% These packages include nice commands from AMS-LaTeX
\usepackage{amssymb,amsmath,amsthm}

% Make the space between lines slightly more
% generous than normal single spacing, but compensate
% so that the spacing between rows of matrices still
% looks normal.  Note that 1.1=1/.9090909...
\renewcommand{\baselinestretch}{1.1}
\renewcommand{\arraystretch}{.91}

% Define an environment for exercises.
\newenvironment{exercise}[1]{\vspace{.1in}\noindent\textbf{Exercise #1 \hspace{.05em}}}{}

% define shortcut commands for commonly used symbols
\newcommand{\R}{\mathbb{R}}
\newcommand{\C}{\mathbb{C}}
\newcommand{\Z}{\mathbb{Z}}
\newcommand{\Q}{\mathbb{Q}}
\newcommand{\N}{\mathbb{N}}

\renewcommand{\mod}[3]{#1 \equiv #2 \pmod{#3}}


%%%%%%%%%%%%%%%%%%%%%%%%%%%%%%%%%%%%%%%%%%

\begin{document}

% If you use Overleaf, the name of the project will be determined by
% what you enter as the document title.
\title{Math 290 Homework Template}

\begin{flushright}
\textsc{Lucas Castro}  \\
Math 290 Sec 01\\
Date 02/01/2023
\end{flushright}

\begin{center}
\textsf{Assignment HW 10} \\
\textsf{Exercises: 10E}
\end{center}

%%%%%%%%%%%%%%%%%%%%%%%%%%%%%%%%%%%%%%%%

\begin{exercise}{10.1}
For each element and set listed below, explain why the element does or does not belong to the set.

(a) Is $3 \in \{1,2,3,4,5,6,7\}$?

(b) Is $\pi \in \{1,2,3,4,5,6,7\}$?

(c) Is $\pi \in \R$?

(d) Is $2/3 \in \{x \in \R: x < 1\}$?

(e) Is $2/3 \in \{x \in \Z: x < 1\}$?
\end{exercise}

\begin{proof}[(a)]
$3$ belongs to $\{1,2,3,4,5,6,7\}$, because it is listed as an element in the set.
\end{proof}

\begin{proof}[(b)]
$\pi$ does not belongs to $\{1,2,3,4,5,6,7\}$, because it is not listed as an element in the set.
\end{proof}

\begin{proof}[(c)]
$\pi$ belongs to $\R$, because $\pi \in \R - \Q$. And $(\R - \Q) \subset \R$. 
\end{proof}

\begin{proof}[(d)]
$2/3 \in \{x \in \R: x < 1\}$, because $2/3 \in \Q$, $\Q \subset \R$, and $2/3 = 0.\overline{6} < 1$. 
\end{proof}

\begin{proof}[(e)]
$2/3 \notin \{x \in \Z: x < 1\}$, because $2/3 \notin \Z$. 
\end{proof}


%%%%%%%%%%%%%%%%%%%%%%%%%%%%%%%%%%%%%%%%

\begin{exercise}{10.2}
Fix
\[A = \{(x,y) \in \Z \times \Z : 4|(x - y)\}\]
and fix
\[B=\{(x,y) \in \Z \times \Z : x \text{ and } y \text{ have the same parity}\}\]
Prove $A \subseteq B$.
\end{exercise}

\begin{proof}
Assume $(a,b) \in A$. Thus $(a,b) \in \Z \times \Z$ and $4|(a-b)$. We can write $(a-b) = 4k$ for some $k \in \Z$. We have two cases to consider.

\textbf{Case 1}: Suppose $a$ and $b$ are even. Then $a = 2m$ and $b = 2n$ for some $m,n \in \Z$. We have
\[(a-b)=(2m - 2n) = 2(m - n) = 4k = 2(2k)\] 
Satisfying the conditions for both $A$ and $B$.

\textbf{Case 2}: Suppose $a$ and $b$ are odd. Then $a = 2m + 1$ and $b = 2n + 1$ for some $m,n \in \Z$. We have
\[(a-b)=(2m + 1) - (2n + 1) = 2m - 2n = 2(m - n) = 4k = 2(2k)\]
Satisfying the conditions for both $A$ and $B$.

In all cases, we have demonstrated that $(a,b)$ satisfies all the properties to belong to B, so $(a,b) \in B$.

Since $(a,b)$ was arbitrary, we have now shown $A \subseteq B$.
\end{proof}


%%%%%%%%%%%%%%%%%%%%%%%%%%%%%%%%%%%%%%%%

\begin{exercise}{10.3}
Let $X$ be the set of integers which are congruent to $-1$ modulo 6 and let $Y$ be the set of integers which are congruent to $2$ modulo $3$. Prove $X \subseteq Y$.
\end{exercise}

\begin{proof}[Set builder notation]
\[X = \{n \in \Z : \mod{n}{-1}{6}\}\]
\[Y = \{n \in \Z : \mod{n}{2}{3}\}\]
\end{proof}

\begin{proof}
Assume $a \in X$. Thus $a \in \Z$ and $\mod{a}{-1}{6}$ (or $6|(a - (-1))$). We can write $(a + 1) = 6k$ for some $k \in \Z$. Therefore, $a = 6k - 1$. 

We now show that $a$ satisfies the all conditions for $Y$, namely $a \in \Z$ (which is true), and $\mod{a}{2}{3}$. We can write $3|(a-2)$, hence $(a - 2) = 3l$ for some $l \in \Z$. We have
\begin{eqnarray*}
(a - 2) &=& 6k - 1 - 2 = 6k - 3 \\
        &=& 3(2k - 1) = 3l
\end{eqnarray*}
This shows that $3|(a-2)$ ($\mod{a}{2}{3}$). Therefore, $a$ satisfies all properties of both $X$ and $Y$. Since $a$ was arbitrary, we have now shown that $X \subseteq Y$.
\end{proof}


%%%%%%%%%%%%%%%%%%%%%%%%%%%%%%%%%%%%%%%%

\begin{exercise}{10.4}
Let $A$ and $B$ be sets inside some universal set $U$.

(a) Prove that $\overline{A \cap B} \subseteq \overline{A} \cup \overline{B}$.

(b) Prove that $\overline{A} \cup \overline{B} \subseteq \overline{A \cap B}$.

(c) Putting those two previous parts together, what have you proved?
\end{exercise}

\begin{proof}[(a) Proof]
Assume $x \in \overline{A \cap B}$. Thus $\lnot (x \in A \cap B)$. In other words $\lnot (x \in A \text{ and } x \in B)$. Using De Morgan's law, we have $\lnot (x \in A)$ or $\lnot (x \in B)$. In other words $x \in \overline{A}$ or $x \in \overline{B}$. Using the definition of union, we have $\overline{A} \cup \overline{B}$. Therefore, $\overline{A \cap B} \subseteq \overline{A} \cup \overline{B}$.
\end{proof}

\begin{proof}[(b) Proof]
Assume $x \in \overline{A} \cup \overline{B}$. Thus $\lnot (x \in A)$ or $\lnot (x \in B)$. Using De Morgan's law, we have $\lnot (x \in A \text{ and } x \in B)$. Using the definition of intersection, we have $x \in \overline{A \cap B}$. Therefore, $\overline{A} \cup \overline{B} \subseteq \overline{A \cap B}$.
\end{proof}

\begin{proof}[(c) Answer]
From proofs (a) and (c) we can conclude that $\overline{A} \cup \overline{B} = \overline{A \cap B}$.
\end{proof}


%%%%%%%%%%%%%%%%%%%%%%%%%%%%%%%%%%%%%%%%

\begin{exercise}{10.5}
Let $X$ and $Y$ be sets. Prove $X - (X - Y) \subseteq X \cap Y$. (Hint: Remember that $s \in S - T$ means $s \in S$ and $s \notin T$. Thus, $s \notin S - T$ means $s \notin S$ or $s \in T$.)
\end{exercise}

\begin{proof}
Assume $a \in X - (X - Y)$. Thus $a \in X$ and $a \notin X - Y$. We can write $a \notin X - Y$ as $a \notin X$ or $a \in Y$. We have already established that $a \in X$, therefore we must conclude that $a \in Y$. We now have $a \in X$ and $a \in Y$. By definition, $a \in X$ and $a \in Y$ is written as $a \in X \cap Y$. Since $a$ was arbitrary, we have now shown that $X - (X - Y) \subseteq X \cap Y$.
\end{proof}


%%%%%%%%%%%%%%%%%%%%%%%%%%%%%%%%%%%%%%%%

\begin{exercise}{10.6}
Given a set $X$, prove that $X \cup \emptyset = X$. (Hint: If you have a case where $x \in \emptyset$, then you know that case doesn't actually happen.)
\end{exercise}

\begin{proof}
We first prove the inclusion $X \cup \emptyset \subseteq X$. Assume $a \in X \cup \emptyset$. Therefore $a \in X$ or $a \in \emptyset$. Since we know that $a \in \emptyset$ never happens, we consider only $a \in X$. Therefore, by definition, $a \in X \cup \emptyset$. Since $a$ was arbitrary, we have now shown that $X \cup \emptyset \subseteq X$.

Conversely, we now show that $X \subseteq X \cup \emptyset$. Assume $b \in X$. By definition, $b \in X \cup \emptyset$ can be expressed as $b \in X$ or $b \in \emptyset$. Since we know that $b \in \emptyset$ never happens, we conclude $b \in X$. Since $b$ was arbitrary, we have now shown that $X \subseteq X \cup \emptyset$.

Since $X \cup \emptyset \subseteq X$ and $X \subseteq X \cup \emptyset$, we have proven that $X \cup \emptyset = X$.
\end{proof}


%%%%%%%%%%%%%%%%%%%%%%%%%%%%%%%%%%%%%%%%

\begin{exercise}{10.7}
Let $n \in \Z$. Prove that
\[\{x \in \Z:n|x\} = \{x \in \Z : \mod{x}{0}{n}\}\]
\end{exercise}

\begin{proof}
We first prove the inclusion $\{x \in \Z:n|x\} \subseteq \{x \in \Z : \mod{x}{0}{n}\}$. Assume $a \in \{x \in \Z:n|x\}$. Thus, $a \in \Z$ and $n|a$. Hence, $a = nk$ for some $k \in \Z$. We can write \\ $\mod{a}{0}{n}$ as $n|(a - 0)$. Thus $(a-0) = nl$ for some $l \in \Z$. We have 
\[(a-0) = (nk - 0) = nk = nl\] 
Therefore, an arbitrary element $a$ satisfies all conditions in both sets.

Now we show, conversely, that $\{x \in \Z : \mod{x}{0}{n}\} \subseteq \{x \in \Z:n|x\}$. Assume $b \in \{x \in \Z : \mod{x}{0}{n}\}$. Thus, $b \in \Z$ and $\mod{b}{0}{n}$. We rewrite $\mod{b}{0}{n}$ as $n|(b - 0)$. So, $b - 0 = b = nk$ for some $k \in \Z$. We have that $n|b$ is defined as $b = nl$ for some $l \in \Z$. Therefore, an arbitrary element $b$ satisfies all conditions in both sets.

Since 
\[\{x \in \Z:n|x\} \subseteq \{x \in \Z : \mod{x}{0}{n}\} \text{ and } \{x \in \Z : \mod{x}{0}{n}\} \subseteq \{x \in \Z:n|x\}\]
we conclude that $\{x \in \Z:n|x\} = \{x \in \Z : \mod{x}{0}{n}\}$.
\end{proof}


%%%%%%%%%%%%%%%%%%%%%%%%%%%%%%%%%%%%%%%%

\begin{exercise}{10.8}
Let $A$, $B$, and $C$ be sets. Prove that
\[A - (B \cap C) \subseteq (A - B) \cup (A - C)\]
\end{exercise}

\begin{proof}
Let $n \in A - (B \cap C)$ for an arbitrary $n$. By definition of set difference, $n \in A$ and $n \notin B \cap C$. In other words, $n \in A$ and $n \notin B$ or $n \notin C$. We consider each case.

\textbf{Case 1}: Suppose $n \notin B$. By the set difference definition we have $n \in A - B$ (since $n \in A)$. Therefore $n \in (A - B) \cup (A - C)$

\textbf{Case 2}: Suppose $n \notin C$. By the set difference definition we have $n \in A - C$ (since $n \in A)$. Therefore $n \in (A - B) \cup (A - C)$

In all cases $n \in (A - B) \cup (A - C)$.

Since $n$ is arbitrary, we have shown that for any element $n \in A - (B \cap C)$, it must also be in $n \in (A - B) \cup (A - C)$. Therefore $A - (B \cap C) \subseteq (A - B) \cup (A - C)$.
\end{proof}


%%%%%%%%%%%%%%%%%%%%%%%%%%%%%%%%%%%%%%%%

\begin{exercise}{10.9}
For each $n \in \N$, define $S_n = \{m \in \Z: m \leq n\}$. Prove that
\[\bigcup_{n \in \N} S_n = \Z\]
(Recall that by Definition 2.9,
\[x \in \bigcup_{i \in I} S_i\]
means that $x \in S_i$ for some $i \in I$.)
\end{exercise}

\begin{proof}
Let $a$ be an arbitrary element of $\Z$. For any $a \in Z$, there exists a natural number $n$ such that $a \leq n$. Therefore for any $a \in \Z$, $a \in S_n$ for some natural number $n$. Since $a$ is arbitrary, we have shown that for any element $a \in \Z$, it must also be in $S_n$ for some natural number $n$. Therefore, $\Z \subseteq \bigcup_{n \in \N} S_n$.

Conversely, for any natural number $n$, $S_n \subseteq \Z$. Therefore, $\bigcup_{n \in \N} S_n \subseteq \Z$.

Therefore,
\[\bigcup_{n \in \N} S_n = \Z\]

\end{proof}


%---------------------------------
% Don't change anything below here
%---------------------------------


\end{document}