\documentclass[12pt,oneside]{article}

% This package simply sets the margins to be 1 inch.
\usepackage[margin=1in]{geometry}

% These packages include nice commands from AMS-LaTeX
\usepackage{amssymb,amsmath,amsthm}

% Make the space between lines slightly more
% generous than normal single spacing, but compensate
% so that the spacing between rows of matrices still
% looks normal.  Note that 1.1=1/.9090909...
\renewcommand{\baselinestretch}{1.1}
\renewcommand{\arraystretch}{.91}

% Define an environment for exercises.
\newenvironment{exercise}[1]{\vspace{.1in}\noindent\textbf{Exercise #1 \hspace{.05em}}}{}

% define shortcut commands for commonly used symbols
\newcommand{\R}{\mathbb{R}}
\newcommand{\C}{\mathbb{C}}
\newcommand{\Z}{\mathbb{Z}}
\newcommand{\Q}{\mathbb{Q}}
\newcommand{\N}{\mathbb{N}}


%%%%%%%%%%%%%%%%%%%%%%%%%%%%%%%%%%%%%%%%%%

\begin{document}

% If you use Overleaf, the name of the project will be determined by
% what you enter as the document title.
\title{Math 290 Homework 06}

\begin{flushright}
\textsc{Lucas Castro}  \\
Math 290 Sec 01\\
Date 01/23/2023
\end{flushright}

\begin{center}
\textsf{Assignment HW 6} \\
\textsf{Exercises: Section 6E}
\end{center}

%%%%%%%%%%%%%%%%%%%%%%%%%%%%%%%%%%%%%%%%

\begin{exercise}{6.1}
Prove that if $x \neq 3$, then $x^2 - 2x + 3 \neq 0$. (Would this result be true if we took $x \in \C$?)
\end{exercise}

\begin{proof}
The implication is trivially true for $x \in \R$, since $x^2 - 2x + 3 = 0$ has no solution in $\R$.
\end{proof}


%%%%%%%%%%%%%%%%%%%%%%%%%%%%%%%%%%%%%%%%

\begin{exercise}{6.2}
Let $n \in \N$. Prove that if $2<n<3$, then $7n + 3$ is odd. 
\end{exercise}

\begin{proof}
The premise is impossible, so the implication is vacuously true.
\end{proof}


%%%%%%%%%%%%%%%%%%%%%%%%%%%%%%%%%%%%%%%%

\begin{exercise}{6.3}
Prove that if $x$ is an odd integer, then $x^2$ is odd.
\end{exercise}

\begin{proof}
Let $x \in \Z$. We work directly. Assume $x$ is odd. Thus $x = 2k + 1$ for some $k \in \Z$. We find
\begin{eqnarray*}
x^2  &=& (2k + 1)^2 = 4k^2 + 4k + 1 \\
&=& 2(2k^2 + 2k) + 1
\end{eqnarray*}
Since $2k^2 + 2k \in \Z$, we have $x^2$ is odd.
\end{proof}

%%%%%%%%%%%%%%%%%%%%%%%%%%%%%%%%%%%%%%%%

\begin{exercise}{6.4}
Prove that if $x$ is an even integer, then $7x - 5$ is odd.
\end{exercise}

\begin{proof}
Let $x \in \Z$. We work directly. Assume $x$ is even. Hence $x = 2k$ for some $k \in \Z$. We find
\begin{eqnarray*}
7x - 5  &=& 7(2k) - 5 = 14k - 5 \\
&=& 14k - 6 + 1 \\ 
&=& 2(7k - 3) + 1 
\end{eqnarray*}
Since $7k - 3 \in \Z$, we have $7x - 5$ is odd.
\end{proof}

\newpage

%%%%%%%%%%%%%%%%%%%%%%%%%%%%%%%%%%%%%%%%

\begin{exercise}{6.5}
Let $a,b,c \in \Z$. Prove that if $a \text{ and } c$ are odd, then $ab + bc$ is even.
\end{exercise}

\begin{proof}
Let $a,c \in \Z$. We work directly. Assume $a \text{ and } c$ are odd. Hence $a = 2k + 1 \text{ and } c = 2i + 1$ for some $k \in \Z$. We find
\begin{eqnarray*}
ab - bc  &=& b(2k + 1) - b(2i + 1) = 2kb + b - 2ib - b \\
&=& 2kb - 2ib = 2(kb - ib)
\end{eqnarray*}
Since $kb - ib \in \Z$, we have $ab - bc$ is even.
\end{proof}

%%%%%%%%%%%%%%%%%%%%%%%%%%%%%%%%%%%%%%%%

\begin{exercise}{6.6}
Let $n \in \Z$. Prove that if $|n| < 1$, then $3n - 2$ is an even integer.
\end{exercise}

\begin{proof}
Let $n \in \Z$. We work directly. Assume $|n| < 1$. Hence $n = 0$. We find
\begin{eqnarray*}
3n - 2 &=& 3(0) - 2 = 0 - 2 \\
&=& - 2 = 2(-1)
\end{eqnarray*}
Since $-1 \in \Z$, we have $3n - 2$ is even.
\end{proof}

%%%%%%%%%%%%%%%%%%%%%%%%%%%%%%%%%%%%%%%%

\begin{exercise}{6.7}
Prove that every odd integer is a difference of two squares of integers. (Hint: Try small cases; write 1, 3, 5, and 7 as differences of squares. It might help to rephrase this statement as an implication, with a premise and conclusion.) \\
In other words: if $x,y \in \Z$ and $x$ is odd, then $x = (y + 1)^2 - y^2$
\end{exercise}

\begin{proof}
Let $x \in \Z$. We work directly. Assume $x$ is odd. Hence $x = 2k + 1$. We find
\begin{eqnarray*}
x &=& (y + 1)^2 - y^2 \\
2k + 1 &=& (y + 1)^2 - y^2 \\
2k + 1 &=& y^2 + 2y + 1 - y^2 \\
2k + 1 &=& 2y + 1
\end{eqnarray*}
Since $y \in \Z$, we have $(y + 1)^2 - y^2$ is odd.
\end{proof}

%%%%%%%%%%%%%%%%%%%%%%%%%%%%%%%%%%%%%%%%

\newpage

\begin{center}
\textsf{Exercise: LaTeX Assignment 1}
\end{center}

The Cartesian product (or simply the product) $A \times B$ of two sets $A$ and $B$ is the set consisting of all ordered pairs whose first coordinate belongs to $A$ and whose second coordinate belongs to B. In other words,

\begin{center}
    $A \times B = \{(a,b) : a \in A \text{ and } b \in B\}$.
\end{center}

\begin{flushleft}
For example if $A=\{x,y\}$ and $B=\{1,2,3\}$, then
\end{flushleft}

\begin{center}
    $A \times B = \{(x,1),(x,2),(x,3),(y,1),(y,2),(y,3)\}$;
\end{center}

\begin{flushleft}
while
\end{flushleft}

\begin{center}
    $B \times A = \{(1,x),(1,y),(2,x),(2,y),(3,x),(3,y)\}$;
\end{center}

\begin{flushleft}
Since, for example, $(x,1) \in A \times B$ and $(x,1) \notin B \times A$, these two sets do not contain the same elements; so $A \times B \neq B \times A$. If $A = \emptyset$ or $B = \emptyset$, then $A \times B = \emptyset$
\end{flushleft}

For the sets $A$ and $B$ just mentioned, $|A| = 2$ and $|B| = 3$; while $|A \times B| = |B \times A| = 6$. Indeed, for all finite sets $A$ and $B$,

\begin{center}
    $|A \times B| = |A| \cdot |B|$.
\end{center}


%---------------------------------
% Don't change anything below here
%---------------------------------


\end{document}