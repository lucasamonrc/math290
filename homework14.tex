\documentclass[12pt,oneside]{article}

% This package simply sets the margins to be 1 inch.
\usepackage[margin=1in]{geometry}

% These packages include nice commands from AMS-LaTeX
\usepackage{amssymb,amsmath,amsthm}

% Make the space between lines slightly more
% generous than normal single spacing, but compensate
% so that the spacing between rows of matrices still
% looks normal.  Note that 1.1=1/.9090909...
\renewcommand{\baselinestretch}{1.1}
\renewcommand{\arraystretch}{.91}

% Define an environment for exercises.
\newenvironment{exercise}[1]{\vspace{.1in}\noindent\textbf{Exercise #1 \hspace{.05em}}}{}

% define shortcut commands for commonly used symbols
\newcommand{\R}{\mathbb{R}}
\newcommand{\C}{\mathbb{C}}
\newcommand{\Z}{\mathbb{Z}}
\newcommand{\Q}{\mathbb{Q}}
\newcommand{\N}{\mathbb{N}}

\renewcommand{\mod}[3]{#1 \equiv #2 \pmod{#3}}


%%%%%%%%%%%%%%%%%%%%%%%%%%%%%%%%%%%%%%%%%%

\begin{document}

% If you use Overleaf, the name of the project will be determined by
% what you enter as the document title.
\title{Math 290 Homework Template}

\begin{flushright}
\textsc{Lucas Castro}  \\
Math 290 Sec 01\\
Date 02/15/2023
\end{flushright}

\begin{center}
\textsf{Assignment 14} \\
\textsf{Exercises: 14.D}
\end{center}

%%%%%%%%%%%%%%%%%%%%%%%%%%%%%%%%%%%%%%%%

\begin{exercise}{14.1}
Prove that $n! > 3^n$ for each natural number $n > 6$.
\end{exercise}

\begin{proof}
Let $P(n)$ be the open sentence
\[P(n): n! > 3^n\]
We work by induction to prove that $P(n)$ is true for each $n \in \N$ with $n > 6$.

\textbf{Base case:} Note that $7! = 5040 > 2187 = 3^7$. Hence, $P(7)$ is true.

\textbf{Inductive step}: Let $k \geq 7$, or $k > 6$, be a natural number, and assume $P(k)$ is true. Then $k! > 3^k$. We wish to show that $(k+1)! > 3^{k+1}$. We find
\begin{eqnarray*}
(k+1)! &=& (k+1)k! \\
&>& (k+1)3^k \\
&>& 3\cdot3^k \\
&=& 3^{k+1}
\end{eqnarray*}
Hence, $P(k+1)$ is true.

Therefore, by induction $P(n)$ is true for each natural number $n > 6$.
\end{proof}


%%%%%%%%%%%%%%%%%%%%%%%%%%%%%%%%%%%%%%%%

\begin{exercise}{14.2}
Prove that if $n$ is any natural number greater than 5, then $n! > n^3$.
\end{exercise}

\begin{proof}
Let $P(n)$ be the open sentence
\[P(n): n! > n^3\]
We work by induction to prove that $P(n)$ is true for each $n \in \N$ with $n > 5$.

\textbf{Base case:} Note that $6! = 720 > 216 = 6^3$. Hence, $P(6)$ is true.

\textbf{Inductive step}: Let $k \geq 6$, or $n > 5$, be a natural number, and assume $P(k)$ is true. Then $k! > k^3$. We wish to show that $(k+1)! > (k+1)^3$. We find
\begin{eqnarray*}
(k+1)! &=& (k+1)k! \\
&>& (k+1)k^3 \\
&>&  7 \cdot k^3 \\
\end{eqnarray*}
We proceed by showing that $7k^3 > k^3 + 3k^2 + 3k + 1 = (k+1)^3$.
\begin{eqnarray*}
7k^3 &>& k^3 + 3k^2 + 3k + 1 \\
7k^3 - k^3 &>& 3k^2 + 3k + 1 \\
6k^3 &>& 3k^2 + 3k + 1 \\
\frac{6k^3}{k^2} &>& \frac{3k^2 + 3k + 1}{k^2} \\
6k &>& 3 + \frac{3}{k} + \frac{1}{k^2}
\end{eqnarray*}
Since $6k > 3 + \frac{3}{k} + \frac{1}{k^2}$ is true for all $k \geq 6$, we can conclude
\[(k+1)k! > 7k^3 > (k+1)^3\]
Hence, $P(k+1)$ is true.

Therefore, by induction $P(n)$ is true for each natural number $n > 5$.
\end{proof}


%%%%%%%%%%%%%%%%%%%%%%%%%%%%%%%%%%%%%%%%

\begin{exercise}{14.3}
Prove that for each $n \in \N$, we have $3^n \geq n^3$.
\end{exercise}

\begin{proof}
Let $P(n)$ be the open sentence
\[P(n): 3^n \geq n^3\]
When $n = 1$, we have $3^1=3\geq1^3 = 1$. When $n = 2$, we have $3^2 = 9 \geq 2^3 = 8$. When $n = 3$, we have $3^3 = 27 \geq 3^3 = 27$. We will now use induction to prove that $P(n)$ is true for all $n \geq 3$.

\textbf{Base case:} $P(3)$ has already been shown to be true.

\textbf{Inductive step:} Now assume $P(k)$, for some integer $k \geq 3$. Hence, we know that $3^k > k^3$. Then
\begin{eqnarray*}
3^{k+1} &=& 3\cdot3^k \\
&\geq& 3 \cdot k^3 \\
&=& k^3 + k^3 + k^3 \\
&\geq& k^3 + k^3 + 3k \\
&\geq& k^3 + 3k^2 + 3k \\ 
&\geq& k^3 + 3k^2 + 3k + 1\\ 
&=& (k+1)^3
\end{eqnarray*}
Hence, $P(k+1)$ is true.

Therefore, by the principle of mathematical induction, $P(n)$ is true for each $n \geq 3$. Since we have already demonstrated $P(1)$ and $P(2)$, we see that $P(n)$ is true for each $n \in \N$.
\end{proof}


%%%%%%%%%%%%%%%%%%%%%%%%%%%%%%%%%%%%%%%%

\begin{exercise}{14.4}
Prove that for any $n \in \N$ with $n \geq 2$, if $P_1,...P_n$ are statements then
\[\lnot(P_1 \land ... \land P_n) \equiv (\lnot P_1) \lor ... \lor (\lnot P_n)\]
\end{exercise}

\begin{proof}
Let $Q(n)$ be the open sentence
\[Q(n): \lnot(P_1 \land ... \land P_n) \equiv (\lnot P_1) \lor ... \lor (\lnot P_n)\]

We will now work by induction on $n \geq 2$.

\textbf{Base case:} $Q(2)$ is just the statement $\lnot(P_1 \land P_2) \equiv (\lnot P_1) \lor (\lnot P_2)$, which is true according to De Morgan's law.

\textbf{Inductive step:} Let $k \in \N$ and assume that $Q(k)$ is true; i.e.,
\[\lnot(P_1 \land ... \land P_k) \equiv (\lnot P_1) \lor ... \lor (\lnot P_k)\]
Then we have
\begin{eqnarray*}
\lnot(P_1 \land ... \land P_{k+1}) &\equiv& (\lnot P_1 \lor ... \lnot P_k) \lor (\lnot P_{k+1}) \\
&\equiv& \lnot(P_1 \land ... \land P_k) \lor (\lnot P_{k+1}) \\
&\equiv& \lnot(P_1 \land ... \land P_k \land P_{k+1})
\end{eqnarray*}
Hence, $Q(k+1)$ is true.

Therefore, $Q(n)$ is true for all $n \in \N$ with $n \geq 2$.
\end{proof}


%%%%%%%%%%%%%%%%%%%%%%%%%%%%%%%%%%%%%%%%

\begin{exercise}{14.5}
Prove that for any $n \in \N$, if $x_1,...,x_n \in \R$, then
\[\Bigg|\sum_{i=1}^{n} x_i\Bigg| \leq \sum_{i=1}^{n}|x_i|\]
\end{exercise}

\begin{proof}
Let $P(n)$ be the open sentence:
\[P(n): \Bigg|\sum_{i=1}^{n} x_i\Bigg| \leq \sum_{i=1}^{n}|x_i|\]
We work by induction to prove that $P(n)$ is true for each natural number $n$.

\textbf{Base case:} $P(2)$ is true, we have
\[\Bigg|\sum_{i=1}^{2} x_i\Bigg| = |x_1 + x_2| \leq |x_1| + |x_2| = \sum_{i=1}^{2}|x_i|\]

\textbf{Inductive step}: Assume $P(k)$. So
\[\Bigg|\sum_{i=1}^{k} x_i\Bigg| = |x_1 + ... + x_k| \leq |x_1| + ... + |x_k| = \sum_{i=1}^{k}|x_i|\]
is true. We want to show $P(k+1)$
\[\Bigg|\sum_{i=1}^{k+1} x_i\Bigg| = |x_1 + ... + x_{k+1}| \leq |x_1| + ... + |x_{k+1}| = \sum_{i=1}^{k+1}|x_i|\]
is true. We start
\begin{eqnarray*}
\Bigg|\sum_{i=1}^{k+1} x_i\Bigg| &=& |(x_1 + ... + x_k) + x_{k+1}| \\
&\leq& |x_1 + ... + x_k| + |x_{k+1}| \\
&\leq& |x_1| + ... + |x_k| + |x_{k+1}| \\
&=& \sum_{i=1}^{k+1}|x_i|
\end{eqnarray*}
Hence, $P(k+1)$ is true.

Therefore, $P(n)$ is true for all $n \in \N$.
\end{proof}


%%%%%%%%%%%%%%%%%%%%%%%%%%%%%%%%%%%%%%%%

\begin{exercise}{14.6}
The \textit{Fibonacci numbers} are a collection of natural numbers labeled $F_1,F_2,F_3,...$ and defined by the rule

\[F_1 = F_2 = 1\]
and for $n > 2$.
\[F_n = F_{n-1} + F_{n-2}\]
For instance, $F_3 = F_2 + F_1 = 2$ and $F_4 = F_3 + F_2 = 3$.

(a) Write down the first fifteen Fibonacci numbers.

(b) Prove by induction that for each $n \geq 1$,
\[\sum_{i=1}^{n}F_i = F_{n+2} - 1\]

(c) Prove by induction that for each $n \geq 1$,
\[\sum_{i=1}^{n}F_i^2 = F_nF_{n+1}\]

\end{exercise}

\begin{proof}[(a)]
Ordered as $F_1, F_2, ..., F_{15}$

\[1,1,2,3,5,8,13,21,34,55,89,144,233,377,610\]
\end{proof}

\begin{proof}[Proof (b)]
Let $P(n)$ be the open sentence:
\[P(n): \sum_{i=1}^{n}F_i = F_{n+2} - 1\]
We work by induction to prove that $P(n)$ is true for each natural number $n \geq 1$.

\textbf{Base case:} $P(1)$ is true since we have
\[\sum_{i=1}^{1}F_i = F_1 = F_{1+2} - 1 = F_3 - 1 = 3 - 1 = 1\]
Note that from \textit{item (a)} we know $F_1 = 1$ and $F_3 = 2$.

\textbf{Inductive step:} Now, assume
\[P(k): \sum_{i=1}^{k}F_i = F_{k+2} - 1\]
for some natural number $k \geq 1$. We want to show that
\[P(k+1): \sum_{i=1}^{k+1}F_i = F_{k+3} - 1\]
is true. Starting with the left-hand side, and simplifying with the right-hand side as a target, we find that
\begin{eqnarray*}
\sum_{i=1}^{k+1}F_i &=& \sum_{i=1}^{k}F_i + F_{k+1} \\
&=& (F_{k+2} - 1) + F_{k+1} \\
&=& F_{k+2} + F_{k+1} - 1 \\
\end{eqnarray*}
Since the Fibonacci numbers follow the $F_n = F_{n-1} + F_{n-2}$, if $n = k + 3$ it follows that
\[F_n = F_{k+3} = F_{k+3-1} + F_{k+3-2} = F_{k+2} + F_{k+1}\]
We finish our proof
\begin{eqnarray*}
\sum_{i=1}^{k+1}F_i &=& F_{k+2} + F_{k+1} - 1 \\
&=& F_{k+3} - 1
\end{eqnarray*}
So $P(k+1)$ is true.

Hence by induction, $P(n)$ is true for all $n \in \N$.
\end{proof}

\begin{proof}[Proof (c)]
Let $P(n)$ be the open sentence:
\[P(n): \sum_{i=1}^{n}F_i^2 = F_nF_{n+1}\]
We work by induction to prove that $P(n)$ is true for each natural number $n \geq 1$.

\textbf{Base case:} $P(1)$ is true since we have
\[\sum_{i=1}^{1}F_i^2 = F_1F_{1+1} = F_1F_2 = 1(1) = 1\]
Note that from \textit{item (a)} we know $F_1 = 1$ and $F_2 = 1$.

\textbf{Inductive step:} Now, assume
\[P(k): \sum_{i=1}^{k}F_i^2 = F_kF_{k+1}\]
for some natural number $k \geq 1$. We want to show that
\[P(k+1): \sum_{i=1}^{k+1}F_i^2 = F_{k+1}F_{k+2}\]
is true. Starting with the left-hand side, and simplifying with the right-hand side as a target, we find that
\begin{eqnarray*}
\sum_{i=1}^{k+1}F_i^2  &=& \sum_{i=1}^{k}F_i^2 + F_{k+1}^2 \\
&=& F_kF_{k+1} + F_{k+1}^2 \\
&=& F_{k+1}(F_k + F_{k+1})
\end{eqnarray*}
Since the Fibonacci numbers follow the $F_n = F_{n-1} + F_{n-2}$, if $n = k+2$ it follows that
\[F_n = F_{k+2} = F_{k+2-1} + F_{k+2-2} = F_{k+1} + F_k\]
We finish our proof
\begin{eqnarray*}
\sum_{i=1}^{k+1}F_i^2 &=& F_{k+1}(F_k + F_{k+1}) \\
&=& F_{k+1}F_{k+2}
\end{eqnarray*}
So $P(k+1)$ is true.

Hence by induction, $P(n)$ is true for all $n \in \N$.
\end{proof}

%%%%%%%%%%%%%%%%%%%%%%%%%%%%%%%%%%%%%%%%

\begin{exercise}{14.7}
Using the definition of the Fibonacci numbers from previous problem, prove by induction that for any integer $n>12$ that $F_n > n^2$.
\end{exercise}

\begin{proof}
Let $P(n)$ be the open sentence
\[P(n): F_n > n^2 \text{ and } F_{n-1} > (n-1)^2\]
We work by induction to prove that $P(n)$ is true for each integer $n > 12$.

\textbf{Base case:} From \textbf{Exercise 14.6(a)}, we have $F_{13} = 233$, $F_{14} = 377$. Thus 
\[P(14): F_{14} = 377 > 196 = 14^2 \text{ and } F_{13} = 233 > 169 = 13^2\]
is true.

\textbf{Inductive step:} Let $k \geq 14$, be an integer and assume that $P(k)$ is true. Therefore $F_k > k^2$ and $F_{k-1} > (k-1)^2$. We wish to show that $F_{k+1} > (k+1)^2$. We find
\begin{eqnarray*}
F_{k+1} &=& F_k + F_{k-1} \\
&>& k^2 + (k-1)^2 \\
&=& k^2 + k^2 - 2k + 1 \\
&=& 2k^2 - 2k + 1 \\
&>& k^2 + 2k + 1
\end{eqnarray*}
is true since
\begin{eqnarray*}
2k^2 - 2k + 1 &>& k^2 + 2k + 1 = (k+1)^2 \\
2k^2 - 2k + 1 - k^2 - 2k - 1&>& 0 \\
k^2 - 4k &>& 0 \\
k(k - 4) &>& 0
\end{eqnarray*}
and $k \geq 14$. So $P(k+1)$ is true.

Hence by induction, $P(n)$ is true for all $n \in \N$.
\end{proof}


%---------------------------------
% Don't change anything below here
%---------------------------------


\end{document}