\documentclass[12pt,oneside]{article}

% This package simply sets the margins to be 1 inch.
\usepackage[margin=1in]{geometry}

% These packages include nice commands from AMS-LaTeX
\usepackage{amssymb,amsmath,amsthm}

% Make the space between lines slightly more
% generous than normal single spacing, but compensate
% so that the spacing between rows of matrices still
% looks normal.  Note that 1.1=1/.9090909...
\renewcommand{\baselinestretch}{1.1}
\renewcommand{\arraystretch}{.91}

% Define an environment for exercises.
\newenvironment{exercise}[1]{\vspace{.1in}\noindent\textbf{Exercise #1 \hspace{.05em}}}{}

% define shortcut commands for commonly used symbols
\newcommand{\R}{\mathbb{R}}
\newcommand{\C}{\mathbb{C}}
\newcommand{\Z}{\mathbb{Z}}
\newcommand{\Q}{\mathbb{Q}}
\newcommand{\N}{\mathbb{N}}

\renewcommand{\mod}[3]{#1 \equiv #2 \pmod{#3}}


%%%%%%%%%%%%%%%%%%%%%%%%%%%%%%%%%%%%%%%%%%

\begin{document}

% If you use Overleaf, the name of the project will be determined by
% what you enter as the document title.
\title{Math 290 Homework Template}

\begin{flushright}
\textsc{Lucas Castro}  \\
Math 290 Sec 01\\
Date 02/04/2023
\end{flushright}

\begin{center}
\textsf{Assignment 11} \\
\textsf{Exercises: 11E}
\end{center}

%%%%%%%%%%%%%%%%%%%%%%%%%%%%%%%%%%%%%%%%

\begin{exercise}{11.1}
Prove the following

(a) There exist $a,b \in \Q$ such that $a^b \in \Q$.

(b) There exist $a,b \in \Q$ such that $a^b \in \R - \Q$.

(c) There exist $a,b \in \R - \Q$ such that $a^b \in \R - \Q$.

(d) There exist $a \in \Q$ and $b \in \R - \Q$ such that $a^b \in \Q$.

(e) There exist $a \in \Q$ and $b \in \R - \Q$ such that $a^b \in \R - \Q$.

(f) There exist $a \in \R - \Q$ and $b \in \Q$ such that $a^b \in \Q$.

(g) There exist $a \in \R - \Q$ and $b \in \Q$ such that $a^b \in \R - \Q$.
\end{exercise}

\begin{proof}[Proof (a)]
Fix $a = 2$ and $b = 3$. We see that $a^b = 2^3 = 8$. Since $2,3,8 \in \Q$, there exists $a,b \in \Q$ such that $a^b \in \Q$.
\end{proof}

\begin{proof}[Proof (b)]
Fix $a = 2$ and $b = 1/2$. We see that $a^b = 2^{1/2} = \sqrt{2}$. Since $2,1/2 \in \Q$ and $\sqrt{2} \in \R - \Q$, there exists $a,b \in \Q$ such that $a^b \in \R - \Q$.
\end{proof}

\begin{proof}[Proof (c)]
Fix $a = \sqrt{2}$ and $b = \sqrt{2}$. We see that $a^b = \sqrt{2}^{\sqrt{2}}$. Since $\sqrt{2}, \sqrt{2}^{\sqrt{2}} \in \R - \Q$, there exist $a,b \in \R - \Q$ such that $a^b \in \R - \Q$. 
\end{proof}

\begin{proof}[Proof (d)]
Fix $a = 1$ and $b = \sqrt{2}$. We see that $a^b = 1^{\sqrt{2}} = 1$. Since $1 \in \Q$ and $\sqrt{2} \in \R - \Q$, there exist $a \in \Q$ and $b \in \R - \Q$ such that $a^b \in \Q$.
\end{proof}

\begin{proof}[Proof (e)]
Fix $a = 2$ and $b = \pi$. We see that $a^b = 2^{\pi}$. Since $2 \in \Q$ and $\pi \in \R - \Q$, there exist $a \in \Q$ and $b \in \R - \Q$ such that $a^b \in \R - \Q$.
\end{proof}

\begin{proof}[Proof (f)]
Fix $a = \sqrt{2}$ and $b = 2$. We see that $a^b = \sqrt{2}^2 = 2$. Since $\sqrt{2} \in \R - \Q$ and $2 \in \Q$, there exist $a \in \R - \Q$ and $b \in \Q$ such that $a^b \in \Q$.
\end{proof}

\begin{proof}[Proof (g)]
Fix $a = \sqrt{2}$ and $b = 1$. We see that $a^b = \sqrt{2}^1 = \sqrt{2}$. Since $\sqrt{2} \in \R - \Q$ and $1 \in \Q$, there exist $a \in \R - \Q$ and $b \in \Q$ such that $a^b \in \R - \Q$.
\end{proof}


%%%%%%%%%%%%%%%%%%%%%%%%%%%%%%%%%%%%%%%%

\begin{exercise}{11.2}
Prove or disprove: Given $x \in \Q$ and $y \in \R - \Q$, then $xy \in \R - \Q$.
\end{exercise}

\begin{proof}[Disproof]
Fix $x = 0$. We have $xy = 0y = 0$. Since $0 \in \Q$, $x = 0$ and $y \in \R - \Q$ is a counterexample.

Therefore if $x \in \Q$ and $y \in \R - \Q$, then $xy \in \R - \Q$.
\end{proof}


%%%%%%%%%%%%%%%%%%%%%%%%%%%%%%%%%%%%%%%%

\begin{exercise}{11.3}
Prove or disprove: Let $s \in \Z$. If $6s - 3$ is odd, then $s$ is odd.
\end{exercise}

\begin{proof}[Disproof]
Fix $s = 2$. We have
$6s - 3 = 6(2) - 3 = 12 - 3 = 9$
is odd. However $s = 2$ is even, and therefore it is a counterexample.
\end{proof}


%%%%%%%%%%%%%%%%%%%%%%%%%%%%%%%%%%%%%%%%

\begin{exercise}{11.4}
Prove or disprove: There exists an integer $x$ such that $x^2 + x$ is odd.
\end{exercise}

\begin{proof}[Disproof]
We can disprove the statement above by proving that for every integer $x$, $x^2 + x$ is even. Let $x \in \Z$. We consider two cases

\textbf{Case 1}: Suppose $x$ is even. Thus $x = 2k$ for some $k \in Z$. We have
\[x^2 + x = (2k)^2 + 2k = 4k^2 + 2k = 2(2k^2 + k)\]
is even

\textbf{Case 2}: Suppose $x$ is odd. Thus $x = 2l + 1$ for some $l \in \Z$. We have
\[x^2 + x = (2k + 1)^2 + (2k + 1) = 4k^2 + 4k + 1 + 2k + 1 = 4k^2 + 6k + 2 = 2(2k^2 + 3k + 1)\]
is even.

In all cases, $x^2 + x$ is even; therefore, there is not an integer $x$ such that $x^2 + x$ is odd.
\end{proof}


%%%%%%%%%%%%%%%%%%%%%%%%%%%%%%%%%%%%%%%%

\begin{exercise}{11.5}
Prove or disprove: Given any positive rational number $a$, there is an irrational number $x \in (0,a)$.
\end{exercise}

\begin{proof}
Let $a \in \Q^+$. Assume $a > 0$. We can define $a$ as an arbitrary positive rational number. Fix $x = \frac{a}{\sqrt{2}}$. We have $x < a$, in other words $\frac{a}{\sqrt{2}} < a$. Therefore $0 < \frac{a}{\sqrt{2}} < a$. Since $\frac{a}{\sqrt{2}} \in \R - \Q$, $\frac{a}{\sqrt{2}} \in (0, a)$, and $a$ is an arbitrary positive rational number, we proved that there is at least one irrational number $x \in (0, a)$. 
\end{proof}


%%%%%%%%%%%%%%%%%%%%%%%%%%%%%%%%%%%%%%%%

\begin{exercise}{11.6}
Prove that for any two real numbers $x < y$, there exists a rational number in the interval $(x, y)$. In this proof you may freely use the fact that if two real numbers are more than $1$ apart, then an integer lies between them.
\end{exercise}

\begin{proof}
Let $x,y \in \R$. Assume $x < y$. Thus $0 < y - x$. We can pick a sufficiently large number $n \in \N$ such that $\frac{1}{n} < y - x$. Now, let $k \in \Z$ be the largest integer for which $\frac{k}{n} \leq x$ and $k \geq 0$, and let $z = \frac{k + 1}{n}$. We can conclude the following:
\begin{itemize}
  \item $x < z$, since $k$ is the largest integer for which $\frac{k}{n} \leq x$, therefore $z \not= \frac{k + 1}{n} \leq x$.
  \item $z < y$, since $z = \frac{k}{n} + \frac{1}{n}$ and we know that $\frac{k}{n} \leq x$ and $\frac{1}{n} \leq y - x$.
  \item $z \in \Q$, since $k + 1, n \in \Z$.
\end{itemize}

Therefore, there exists a rational number $z$ in the interval $(x, y)$.
\end{proof}


%---------------------------------
% Don't change anything below here
%---------------------------------


\end{document}