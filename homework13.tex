\documentclass[12pt,oneside]{article}

% This package simply sets the margins to be 1 inch.
\usepackage[margin=1in]{geometry}

% These packages include nice commands from AMS-LaTeX
\usepackage{amssymb,amsmath,amsthm}

% Make the space between lines slightly more
% generous than normal single spacing, but compensate
% so that the spacing between rows of matrices still
% looks normal.  Note that 1.1=1/.9090909...
\renewcommand{\baselinestretch}{1.1}
\renewcommand{\arraystretch}{.91}

% Define an environment for exercises.
\newenvironment{exercise}[1]{\vspace{.1in}\noindent\textbf{Exercise #1 \hspace{.05em}}}{}

% define shortcut commands for commonly used symbols
\newcommand{\R}{\mathbb{R}}
\newcommand{\C}{\mathbb{C}}
\newcommand{\Z}{\mathbb{Z}}
\newcommand{\Q}{\mathbb{Q}}
\newcommand{\N}{\mathbb{N}}

\renewcommand{\mod}[3]{#1 \equiv #2 \pmod{#3}}


%%%%%%%%%%%%%%%%%%%%%%%%%%%%%%%%%%%%%%%%%%

\begin{document}

% If you use Overleaf, the name of the project will be determined by
% what you enter as the document title.
\title{Math 290 Homework Template}

\begin{flushright}
\textsc{Lucas Castro}  \\
Math 290 Sec 01\\
Date 02/11/2023
\end{flushright}

\begin{center}
\textsf{Assignment 13} \\
\textsf{Exercises: Section 13.B}
\end{center}

%%%%%%%%%%%%%%%%%%%%%%%%%%%%%%%%%%%%%%%%

\begin{exercise}{13.1}
Prove that for each $n \in \N$,
\[\sum_{i = 1}^{n} (2i - 1) = n^2\]
\end{exercise}

\begin{proof}
Let $P(n)$ be the open sentence:
\[P(n): \sum_{i = 1}^{n} (2i - 1) = n^2\]
We work by induction to prove that $P(n)$ is true for each $n \in \N$.

\textbf{Base case}: $P(1)$ is true, since we have
\[\sum_{i = 1}^{1} (2i - 1) = (2\cdot1 - 1) = 2 - 1 = 1 = 1^2\]

\textbf{Inductive step}: Let $k \in \N$ and assume that
\[P(k): \sum_{i = 1}^{k} (2i - 1) = k^2\]
is true. We want to show that
\[P(k + 1): \sum_{i = 1}^{k + 1} (2i - 1) = (k+1)^2\]
is true. Starting with the left-hand side, and simplifying with the right-hand side as a target, we find that
\begin{eqnarray*}
\sum_{i = 1}^{k + 1} (2i - 1) &=& 1 + (2\cdot2 - 1) + (2\cdot3 - 1) + ... + (2k - 1) + (2k + 1) \\
&=& \sum_{i = 1}^{k} (2i - 1) + (2k + 1) \\
&=& k^2 + (2k + 1) = k^2 + 2k + 1 = (k + 1)^2
\end{eqnarray*}
So $P(k + 1)$ is true.

Hence, by induction, $P(n)$ is true for all $n \in \N$.
\end{proof}


%%%%%%%%%%%%%%%%%%%%%%%%%%%%%%%%%%%%%%%%

\begin{exercise}{13.2}
Prove that for each $n \in \N$,
\[\sum_{i=1}^{n} \frac{1}{(2i-1)(2i+1)} = \frac{n}{2n + 1}\]
\end{exercise}

\begin{proof}
Let $P(n)$ be the open sentence:
\[P(n): \sum_{i=1}^{n} \frac{1}{(2i-1)(2i+1)} = \frac{n}{2n + 1}\]
We work by induction to prove that $P(n)$ is true for each $n \in \N$.

\textbf{Base case}: $P(1)$ is true, since we have
\[\sum_{i=1}^{1} \frac{1}{(2i-1)(2i+1)} = \frac{1}{(2(1)-1)(2(1)+1)} = \frac{1}{(2-1)(2+1)} = \frac{1}{3} = \frac{1}{2(1) + 1}\]

\textbf{Inductive step}: Let $k \in \N$ and assume that
\[P(k): \sum_{i=1}^{k} \frac{1}{(2i-1)(2i+1)} = \frac{1}{(2k-1)(2k+1)} = \frac{k}{2k + 1}\]
is true. We want to show that
\[P(k+1): \sum_{i=1}^{k+1} \frac{1}{(2i-1)(2i+1)} = \frac{1}{(2(k+1)-1)(2(k+1)+1)} = \frac{k+1}{2(k+1) + 1}\]
is true. First we simplify $P(k+1)$
\[P(k+1): \sum_{i=1}^{k+1} \frac{1}{(2i-1)(2i+1)} = \frac{1}{(2k+1)(2k+3)} = \frac{k+1}{2k+3}\]
Now, starting with the left-hand side, and simplifying with the right-hand side as a target, we find that
\begin{eqnarray*}
\sum_{i=1}^{k+1} \frac{1}{(2i-1)(2i+1)} &=& \sum_{i=1}^{k} \frac{1}{(2i-1)(2i+1)} + \frac{1}{(2k+1)(2k+3)} \\
&=& \frac{k}{2k + 1} + \frac{1}{(2k+1)(2k+3)} = \frac{k(2k+3) + 1}{{(2k+1)(2k+3)}} \\
&=& \frac{2k^2 + 3k + 1}{{(2k+1)(2k+3)}} = \frac{(2k+1)(k+1)}{{(2k+1)(2k+3)}} \\
&=& \frac{k+1}{2k+3}
\end{eqnarray*}
So $P(k + 1)$ is true.

Hence, by induction, $P(n)$ is true for all $n \in \N$.
\end{proof}


%%%%%%%%%%%%%%%%%%%%%%%%%%%%%%%%%%%%%%%%

\begin{exercise}{13.3}
Prove that for each $n \in \N$,
\[\sum_{i=1}^{n} i^2 = \frac{n(n+1)(2n+1)}{6}\]
\end{exercise}

\begin{proof}
Let $P(n)$ be the open sentence:
\[P(n): \sum_{i=1}^{n} i^2 = \frac{n(n+1)(2n+1)}{6}\]
We work by induction to prove that $P(n)$ is true for each $n \in \N$.

\textbf{Base case}: $P(1)$ is true, since we have
\[\sum_{i=1}^{1} i^2 = 1^2 = \frac{1(1+1)(2(1)+1)}{6} = \frac{2(3)}{6} = 1 \]

\textbf{Inductive step}: Let $k \in \N$ and assume that
\[P(k): \sum_{i=1}^{k} i^2 = \frac{k(k+1)(2k+1)}{6}\]
is true. We want to show that
\[P(k+1): \sum_{i=1}^{k+1} i^2 = \frac{(k+1)(k+2)(2k+3)}{6}\]
is true. Starting with the left-hand side, and simplifying with the right-hand side as a target, we find that
\begin{eqnarray*}
\sum_{i=1}^{k+1} i^2 &=& \sum_{i=1}^{k} i^2 + (k+1)^2 =  \frac{k(k+1)(2k+1)}{6} + (k+1)^2 \\
&=& \frac{k(k+1)(2k+1) + 6(k+1)^2}{6} = \frac{(k+1)(k(2k+1) + 6(k + 1))}{6} \\
&=& \frac{(k+1)(k(2k+1) + 6k + 6))}{6} = \frac{(k+1)(2k^2+k+6k+1)}{6} \\ 
&=& \frac{(k+1)(2k^2+7k+1)}{6} = \frac{(k+1)(k+2)(2k+3)}{6} 
\end{eqnarray*}
So $P(k + 1)$ is true.

Hence, by induction, $P(n)$ is true for all $n \in \N$.
\end{proof}


%%%%%%%%%%%%%%%%%%%%%%%%%%%%%%%%%%%%%%%%

\begin{exercise}{13.4}
(a) Prove that for each $n \in \N$,
\[n < 3^n\]

(b) Prove that for each $n \in \Z$, $n < 3^n$.
\end{exercise}

\begin{proof}[Proof (a)]
We work by induction on $n \in \N$.

\textbf{Base case}: We see that $1 < 3^1 = 3$

\textbf{Inductive step}: Let $k \in \N$ and assume $k < 3^k$. We want to prove $k+1 < 3^{k+1}$. We find
\begin{eqnarray*}
3^{k+1} &=& 3 \cdot 3^k \\
&>& 3 \cdot k \\
&=& k + k + k \\
&\geq& k + 1
\end{eqnarray*}
So by induction we know that $k < 3^k$ for each $n \in \N$.
\end{proof}


\begin{proof}[Proof (b)]
Let $k \in \Z$ and assume $k < 3^k$. Since $n \in \N$ for any positive integer $n$, \textit{Proof (a)} shows that $k < 3^k$ where $k \in \Z^+$. We consider the two remaining cases.

\textbf{Case 1}: Suppose $k = 0$. By definition $a^0 = 1$ for any $a \in \R$. Therefore $0 = k < 3^0 = 1$ is true.

\textbf{Case 2}: Suppose $k$ is negative. Since $k < 3^k$ holds for all positive integers, and all negative integers are less than the positive integers, then it follows that $k < 3^k$ for negative integers.

In all cases, $k < 3^k$.
\end{proof}


%%%%%%%%%%%%%%%%%%%%%%%%%%%%%%%%%%%%%%%%

\begin{exercise}{13.5}
Let $x \in \R - \{1\}$. Prove that for each $n \in \N$,
\[\sum_{i = 0}^{n} x^i = \frac{1- x^{n + 1}}{1 - x}\]
\end{exercise}

\begin{proof}
Let $P(n)$ be the open sentence:
\[P(n): \sum_{i = 0}^{n} x^i = \frac{1- x^{n + 1}}{1 - x}\]
We work by induction to prove that $P(n)$ is true for each $n \in \N$.

\textbf{Base case}: $P(1)$ is true, since we have
\begin{eqnarray*}
\sum_{i = 0}^{1} x^i &=& x^0 + x^1 = 1 + x = \frac{1- x^{1 + 1}}{1 - x} \\
&=& \frac{1 - x^{2}}{1 - x} = \frac{(1 - x)(1 + x)}{1 - x} = 1 + x
\end{eqnarray*}

\textbf{Inductive step}: Let $k \in \N$ and assume that
\[P(k): \sum_{i = 0}^{k} x^i = \frac{1- x^{k + 1}}{1 - x}\]
is true. We want to show that
\[P(k + 1): \sum_{i = 0}^{k + 1} x^i = \frac{1- x^{k + 2}}{1 - x}\]
is true. Starting with the left-hand and simplifying with the right-hand side as a target, we find that
\begin{eqnarray*}
\sum_{i = 0}^{k + 1} x^i &=& \sum_{i = 0}^{k} x^i + x^{k+1} = \frac{1- x^{k + 1}}{1 - x} + x^{k+1}\\
&=& \frac{1 - x^{k+1} + x^{k+1}(1-x)}{1-x} = \frac{1 - x^{k+1} + x^{k+1}-x^{k+2}}{1-x} \\
&=& \frac{1 - x^{k+2}}{1-x}
\end{eqnarray*}
Therefore, $P(k+1)$ is true.

Hence, by induction, $P(n)$ is true for all $n \in \N$.
\end{proof}


%%%%%%%%%%%%%%%%%%%%%%%%%%%%%%%%%%%%%%%%

\begin{exercise}{13.6}
Let $x \in \R$ and assume $x > -1$. Prove that for each $n \in \N$,
\[(1 + x)^n \geq 1 + nx\]
\end{exercise}

\begin{proof}
Let $P(n)$ be the open sentence:
\[P(n): (1 + x)^n \geq 1 + nx\]
We work by induction to prove that $P(n)$ is true for each $n \in \N$.

\textbf{Base case}: $P(1)$ is true, since
\[(1+x)^1 = 1 + x \geq 1 + 1x = 1 + x \]

\textbf{Inductive step}: Let $k \in \N$ and assume that
\[P(k): (1 + x)^k \geq 1 + kx\]
is true. We want to show that
\[P(n): (1 + x)^{k+1} \geq 1 + (k+1)x\]
is true. We find
\begin{eqnarray*}
(1+x)^{k+1} &=& (1+x)(1+x)^k \\
&\geq& (1+kx)(1+x) \\
&=& 1+x+kx+kx^2 \\
&=& 1+(k+1)x+kx^2 \\
&\geq& 1+(k+1)x \\
\end{eqnarray*}
Therefore, $P(k+1)$ is true.

Hence, by induction, $P(n)$ is true for each $n \in \N$.

\end{proof}


%%%%%%%%%%%%%%%%%%%%%%%%%%%%%%%%%%%%%%%%

\begin{exercise}{13.7}
Let $S$ be \textit{any} nonempty set of natural numbers. Prove that $S$ has a least element.

The fact that any nonempty subset of the natural numbers has a least element is called the \textit{well-ordering principle}.
\end{exercise}

\begin{proof}
Let $P(n)$ be the open sentence:
\[P(n): \text{Any nonempty subset $S$ of $\N$, with $n$ elements, has a least element.}\]
We work by induction to prove that $P(n)$ is true for each $n \in \N$.

\textbf{Base case}: $P(1)$ is true, since $1 \leq x$ for all $x \in \N$.

\textbf{Inductive step}: Let $k \in \N$ and assume $P(k)$. In other words, assume that every subset nonempty $S$ of $\N$, with $k$ elements, has least element. We want to show that $P(k + 1)$ is true.

Let $S$ be any nonempty subset of $\N$. We pick an element of $S$ and call it $a$. Let $T = S - \{a\}$. We note that $T$ has $k$ elements, from the inductive hypothesis, and therefore $T$ has a least element $b$. By definition, we derive that $b \leq x$ for each $x \in T$. Note that since $b \in T = S - \{a\}$, it follows that $b \neq a$. Thus, we consider two cases.

\textbf{Case 1}: Suppose $b < a$, then $b \leq x$ for each $x \in T \cup \{a\} = S$. Hence, $b$ is the least element of $S$.

\textbf{Case 2}: Suppose $a < b$, then $a < b \leq x$ for each $x \in T$. Therefore, $a \leq x$ for each $x \in T \cup \{a\} = S$. It follows that $a$ is the least element of $S$.

In all cases, $S$ has a least element. Since $S$ was an arbitrary nonempty set with $k + 1$ elements, $P(k+1)$ is true. Hence, by induction, $P(n)$ is true for all $n \in \N$.
\end{proof}


%%%%%%%%%%%%%%%%%%%%%%%%%%%%%%%%%%%%%%%%

\begin{exercise}{13.8}
Prove the following variation of the pigeonhole principle.

Let $m \in \N \cup {0}$, let $n \in \N$, and assume $m < n$. If we suppose $m$ objects are placed in $n$ bins, conclude that some bins do not contain any object.
\end{exercise}

\begin{proof}
Let $P(n)$ be the open sentence

$P(n)$: For each $m \in \N \cup \{0\}$, if $m < n$ and $m$ objects are placed in $n$ bins, then some bins do not contain any object.

\textbf{Base case:} We verify that $P(1)$. If we have one bin and no objects, then the bin is empty.

\textbf{Inductive step:} Let $k \in \N$ and assume $P(k)$. We now prove $P(k+1)$. Let $m < k+1$ and $m$ objects are placed in $k+1$ bins. We must show that one bin is empty. Pick an object $x$, and consider the following cases

\textbf{Case 1.} Suppose that the object $x$ does not shares a bin with another object. In this case each bin will have a single object, and since when $m < k$ there is an empty bin, it follows that there should be at least one empty bin when $n = k+1$.

\textbf{Case 2.} Suppose that object $x$ shares a bin with other objects, let $y$ the number of objects inside that bin. Then there will be $m - (xy)$ remaining objects. $m - (xy) < m$, and most definitely $m - (xy) < k + 1$, therefore there should be at least one empty bin when $n = k+1$
\end{proof}

In both cases, there will be at least one bin empty. Hence, by induction, $P(n)$ is true for all $n \in \N$.


%---------------------------------
% Don't change anything below here
%---------------------------------


\end{document}